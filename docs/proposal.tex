\documentclass{article}
\usepackage[final]{neurips_2019}
\usepackage{subfig}
\usepackage{graphicx}
\usepackage[utf8]{inputenc} 
\usepackage[T1]{fontenc} 
\usepackage{hyperref} 
\usepackage{url} 
\usepackage{booktabs} 
\usepackage{amsfonts} 
\usepackage{nicefrac} 
\usepackage{microtype} 
\usepackage{array} 
\usepackage{tabularx} 
\usepackage{ragged2e} 
\usepackage{booktabs} 
\usepackage{array} 

\bibliography{ref}
\title{Mixture of Expert Model for Code Generation}

\author{
TEAM 11 \\
\textbf{Adithya Kameswara Rao}
\texttt{akameswa@andrew.cmu.edu} \\
\textbf{Santhoshkumar Panneerselvam}
\texttt{spanneer@andrew.cmu.edu} \\
\textbf{Yihao Jia}
\texttt{yihaoj@andrew.cmu.edu} \\
\textbf{Yirui Zhu}
\texttt{yiruiz@andrew.cmu.edu} \\ 
}

\begin{document}
	\maketitle
	\vspace{-1.5em}
	\section{Introduction}

	Deep learning, as a specialized branch within the machine learning domain,
	exhibits an increasingly commanding prowess in content generation. While the consensus
	suggests that complex models tend to achieve superior outcomes, the necessity
	for advanced GPUs and the associated costs render such endeavors viable only
	for behemoth corporations. In this context, the Mixture-of-Experts (MoE) [1]
	method emerges as a pioneer that democratizes deep learning by circumventing
	the prohibitive costs associated with simpler models but equivalent
	performance. Essentially, MoE segments the problem space into discrete sub-regions,
	each addressed by specialized, simpler models or 'experts,' rather than employing
	a monolithic model for the entire domain. With adept integration, such an MoE framework
	can achieve performance parity with larger counterpart models, albeit in a
	more economically feasible manner.

	Taking the realm of code generation as a point of reference, and as depicted in
	Figure 1, we propose a MoE framework that deftly selects the C language expert
	model for output generation if the input code snippet is C language. Analogously,
	when presented with Python code, the system channels the task to the expert model
	versed in Python for processing. This sophisticated architecture empowers the
	system to harness the prowess of distinct expert models tailored to various
	programming languages, thereby significantly enhancing the precision of outputs
	over the conventional baseline model.

	\begin{figure}[htbp]
		\centering
		\includegraphics[width=3.4in]{fig/3.png}
		\caption{Applying MoE Architecture on Code Generation}
		\label{fig:flow3}
	\end{figure}

	In our pursuit, we endeavor to meticulously examine the Mixture-of-Experts (MoE)
	paradigm as it pertains to code generation. Presently, Large Language Models (LLMs)
	tasked with code generation typically function as monolithic systems across a variety
	of programming languages. Although these models are capable of generating
	competent outputs, their efficacy may wane when tasked with specific programming
	languages. For instance, a sophisticated model might exhibit superior performance
	in generating code for Python as opposed to Java and C++, despite all these
	languages falling well within its realm of expertise. Such observations compellingly
	suggest that adopting the MoE framework for code generation can be both intellectually
	sound and potentially advantageous. Given that code generation naturally
	divides into discrete sub-domains—namely, different programming languages—the assignment
	of specialized experts within an MoE model for each language offers a significant
	opportunity to surpass the capabilities of a singular, complex model in
	handling multiple programming languages.

	\section{Literature Review}

	MoE stands as a cutting-edge approach, leveraging a sophisticated model
	ensemble to handle diverse and complex tasks efficiently. The powerful fusion of
	MoE and a code-gen model directs specialized expert models, elevating its
	capabilities in code generation for different programming languages.

	\subsection{Mixture-of-Experts (MoE)}

	\textbf{Mixtral [1]}: The "Mixtral of Experts" paper introduces the Mixtral 8x7B
	model, an advanced sparse MoE language model built upon the Mistral 7B
	architecture. It features eight feedforward blocks per layer and selects two experts
	for each token. Overall, it boasts 47B parameters but utilizes only 13B during
	inference. Thus, this model excels in computational efficiency and performance,
	surpassing benchmarks set by models like GPT-3.5 and Llama 2 70B in areas
	including mathematics, code generation, and multilingual translation.

	\textbf{Gemini [2]}: Gemini 1.5, boasting significantly improved performance and
	greater speed over its predecessor, Gemini 1.0, primarily attributes its advancements
	to the incorporation of the MoE framework into its foundational architecture.
	As claimed by Google, Gemini 1.5 outperforms 87\% in benchmark experiments
	through the integration of the MoE approach within models such as GShard-Transformer,
	Switch-Transformer, M4, among others.

	\textbf{SegMoE [3]}: SegMoE, introduced by Segmind, is an innovative open-source
	framework designed to enhance text-to-image generation. It combines multiple
	generative image models into more comprehensive and efficient systems, utilizing
	Stable Diffusion's architectural principles enhanced with the sparse MoE layers.
	This addition optimizes expert selection for each token, aiming to significantly
	boost image quality and prompt accuracy.

	\subsection{Code Generation Model}

	\textbf{SantaCoder [4]}: SantaCoder is a 1.1 billion parameter decoder-only
	transformer architecture designed for code generation tasks, trained on Java,
	JavaScript, and Python from The Stack dataset. Building upon earlier efforts
	in the BigCode community, the model integrates Multi Query Attention (MQA) and
	Fill-in-the-Middle (FIM) techniques, along with preprocessing filters. After 600,000
	iterations of training, SantaCoder outperforms larger models on code
	generation and infilling tasks across Java, JavaScript, and Python programming
	languages.

	\textbf{Phi-2 [5]}: Phi-2 is a large language model based on the Transformer
	architecture, comprising approximately 2.7 billion trainable parameters. Its
	training data encompassed the same sources utilized for the earlier Phi-1.5
	model, supplemented by an additional corpus consisting of synthetically generated
	natural language processing (NLP) texts and filtered web content. Phi-2
	demonstrated performance nearing the state-of-the-art among models constrained
	to fewer than 13 billion parameters on commonsense reasoning, language comprehension,
	and logical inference capabilities.

	\section{Baseline Model Ideas}

	Our strategy encompasses two primary phases: construction and evaluation. In the
	construction phase, our efforts will focus on 1) Selecting a code generation model
	of exceptional versatility, capable of accommodating a broad spectrum of
	programming languages; 2) Tailoring this model through precise fine-tuning to
	create dedicated experts for each coding language; 3) Integrating these
	specialized models into a cohesive MoE model.

	For the evaluation phase, we will employ the MultiPL-E system [6], which acts
	as a multi-programming language evaluation framework, for model benchmarking.
	Basically, MultiPL-E employs pass rates as a metric, evaluating the
	probability that a model's output will successfully resolve a specified
	problem within a given number of attempts. This metric offers a quantifiable gauge
	of a model's proficiency in producing functionally accurate code across diverse
	programming languages and tasks.

	Upon the conclusion of this project, a thorough analysis will be articulated, encapsulating
	the efficacy of the MoE model in juxtaposition with the conventional universal
	model.

	\section{Dataset Description}
  
	Through preliminary research, our datasets will be drawn from the BigCode [7]
	database, which contains 30 programming languages. The original dataset, initially
	sized at approximately 102TB, underwent filtering processes, including near-deduplication
	and license filtering, resulting in a reduced size of 3TB. The filtering methodology
	follows the approaches outlined in Codex[8] and involves removing files that
	meet the following criteria:

	1. Average line length exceeding 100 characters.\\ 2. Maximum line length
	surpassing 1,000 characters.\\ 3. Alphanumeric character fraction less than 25$\%$.\\
	4. Remove auto-generated files like config, etc.

	\textbf{Dataset Structure}

	Each data instance corresponds to a file, with its content stored in the "content"
	feature.

	\textbf{Data Fields}

	\begin{table}[htbp]
		\centering 
		\begin{tabularx}
			{\textwidth}{|l|l|X|} 
			\hline \textbf{Column Name} & \textbf{Type} & \textbf{Description} \\
			\hline content & string & file content \\ \hline size & integer &
			uncompressed file size\\ \hline lang & string & programming language\\
			\hline avg\_line\_length & float & average line length\\ \hline max\_line\_length
			& integer & maximum line length\\ \hline alphanum\_fraction & float &
			fraction of alphanumeric characters\\ \hline hexsha & string & unique git hash\\
			\hline max\_stars\_repo\_path & string & path in the repository with the
			maximum stars\\ \hline max\_forks\_repo\_path & string & path in the repository
			with the maximum forks\\ \hline max\_issues\_repo\_path & string & path in
			the repository with the maximum issues\\ \hline max\_stars\_repo\_name & string
			& name of the repository with the maximum stars\\ \hline max\_forks\_repo\_name
			& string & name of the repository with the maximum forks\\ \hline max\_issues\_repo\_name
			& string & name of the repository with the maximum issues\\ \hline max\_stars\_repo\_head\_hexsha
			& string & repository head hexsha for repo with maximum stars\\ \hline max\_forks\_repo\_head\_hexsha
			& string & repository head hexsha for repo with maximum forks\\ \hline max\_issues\_repo\_head\_hexsha
			& string & repository head hexsha for repo with max issues\\ \hline max\_stars\_repo\_licenses
			& string & repository licenses for repo with maximum stars\\ \hline max\_forks\_repo\_licenses
			& string & repository licenses for repo with maximum forks\\ \hline max\_issues\_repo\_licenses
			& string & repository licenses for repo with maximum issues\\ \hline max\_stars\_count
			& integer & number of stars in repository with maximum stars\\ \hline max\_forks\_count
			& integer & number of forks in repository with maximum forks\\ \hline max\_issues\_count
			& integer & number of issues in repository with maximum issues\\ \hline
		\end{tabularx}
	\end{table}

	\vspace{-1em}

	For the fine-tuning process, we will select a dataset of 4 programming
	languages based on the compute feasibility and evaluation procedures. Once selected,
	a subset (with a statistically significant sample size) of the dataset for
	respective programming languages will be chosen and worked with.

	\vspace{-1em}
	
  \section*{References}

	\small [1] {\textbf{Mixtral of Experts}}, arXiv:2401.04088

	[2] {\textbf{Gemini 1.5: }}\url{https://blog.google/technology/ai/google-gemini-next-generation-model-february-2024}

	[3] \textbf{Segmind's SegMoE: }\url{https://blog.segmind.com/introducing-segmoe-segmind-mixture-of-diffusion-experts/}

	[4] \textbf{SantaCoder: don't reach for the stars!}, arXiv:2301.03988

	[5] \textbf{Phi-2: }\url{https://huggingface.co/microsoft/phi-2}

	[6] \textbf{MultiPL-E: Scalable and Extensible Approach to Benchmarking Code Generation}
	arXiv:2208.08227

	[7] \textbf{BigCode: }\url{https://www.bigcode-project.org/docs/about/the-stack/}.

	[8] {\textbf{Evaluating Large Language Models Trained on Code}}, arXiv:2107.03374

\end{document}